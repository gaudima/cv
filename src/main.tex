%----------------------------------------------------------------------------------------
%	DOCUMENT DEFINITION
%----------------------------------------------------------------------------------------

% we use article class because we want to fully customize the page
\documentclass[10pt,A4]{article}


%----------------------------------------------------------------------------------------
%	ENCODING
%----------------------------------------------------------------------------------------

%we use utf8 since we want to build from any machine
\usepackage[utf8]{inputenc}

%----------------------------------------------------------------------------------------
%	LOGIC
%----------------------------------------------------------------------------------------

\usepackage{xifthen}
\usepackage{calc}
\usepackage[hidelinks]{hyperref}

%----------------------------------------------------------------------------------------
%	FONT
%----------------------------------------------------------------------------------------

% some tex-live fonts - choose your own

%\usepackage[defaultsans]{droidsans}
%\usepackage[default]{comfortaa}
%\usepackage{cmbright}
\usepackage[default]{raleway}
%\usepackage{fetamont}
%\usepackage[default]{gillius}
%\usepackage[light,math]{iwona}
%\usepackage[thin]{roboto}

% set font default
\renewcommand*\familydefault{\sfdefault}
\usepackage[T1]{fontenc}

% more font size definitions
\usepackage{moresize}

% font icons package
\usepackage{fontawesome}

\usepackage{enumitem}

%----------------------------------------------------------------------------------------
%	PAGE LAYOUT  DEFINITIONS
%----------------------------------------------------------------------------------------

%define page styles using geometry
\usepackage[a4paper]{geometry}

% for example, change the margins to 2 inches all round
\geometry{top=2cm, bottom=2cm, left=2cm, right=2cm}

% use customized header
\usepackage{fancyhdr}
\pagestyle{fancy}

%less space between header and content
\setlength{\headheight}{-5pt}

% customize header entries
\lhead{}
\rhead{}
\chead{}

%indentation is zero
\setlength{\parindent}{0mm}

%----------------------------------------------------------------------------------------
%	GRAPHICS DEFINITIONS
%----------------------------------------------------------------------------------------

% for drawing graphics and charts
\usepackage{graphicx}
\usepackage{tikz}
\usetikzlibrary{shapes, backgrounds}

% use to vertically center content
% credits to: http://tex.stackexchange.com/questions/7219/how-to-vertically-center-two-images-next-to-each-other
\newcommand{\vcenteredinclude}[1]{\begingroup
\setbox0=\hbox{\includegraphics{#1}}%
\parbox{\wd0}{\box0}\endgroup}

% use to vertically center content
% credits to: http://tex.stackexchange.com/questions/7219/how-to-vertically-center-two-images-next-to-each-other
\newcommand*{\vcenteredhbox}[1]{\begingroup
\setbox0=\hbox{#1}\parbox{\wd0}{\box0}\endgroup}

%----------------------------------------------------------------------------------------
%	ICON-SET EMBEDDING
%----------------------------------------------------------------------------------------

% at this point we simplify our icon-embedding by simply referring to a set of png images.
% if you find a good way of including svg without conflicting with other packages you can
% replace this part
\newcommand{\icon}[2]{\colorbox{black}{\makebox(#2, #2){\textcolor{white}{\large\csname fa#1\endcsname}}}}    %icon shortcut
\newcommand{\icontext}[3]{                        %icon with text shortcut
\vcenteredhbox{\icon{#1}{#2}}\hspace{0.2cm}\vcenteredhbox{\textcolor{black}{#3}}
}

%----------------------------------------------------------------------------------------
% 	HEADER
%----------------------------------------------------------------------------------------

% remove top header line
\renewcommand{\headrulewidth}{0pt}

%remove botttom header line
\renewcommand{\footrulewidth}{0pt}

%remove pagenum
\renewcommand{\thepage}{}

%remove section num
\renewcommand{\thesection}{}


%----------------------------------------------------------------------------------------
%
% 	TIKZ GRAPHICS
%
%----------------------------------------------------------------------------------------

\newcounter{a}
\newcounter{b}
\newcounter{c}
\newcounter{barcount}

%----------------------------------------------------------------------------------------
% 	BAR CHART
%----------------------------------------------------------------------------------------

% draw a bar chart
% param 1: width
% param 2: height
% param 3: border color
% param 4: label text color
% param 5: label bg color
% param 6: cat 1 color
\newenvironment{barchart}[8]{

\newcommand{\barwidth}{0.35}
\newcommand{\barsep}{0.2}

% param 1: overall percent
% param 2: label
% param 3: cat 1 percent
% param 4: cat 2 percent
% param 5: cat 3 percent
\newcommand{\baritem}[5]{

\pgfmathparse{##3+##4+##5}
\let\perc\pgfmathresult

\pgfmathparse{#2}
\let\barsize\pgfmathresult

\pgfmathparse{\barsize*##3/100}
\let\barone\pgfmathresult

\pgfmathparse{\barsize*##4/100}
\let\bartwo\pgfmathresult

\pgfmathparse{\barsize*##5/100}
\let\barthree\pgfmathresult

\pgfmathparse{(\barwidth*\thebarcount)+(\barsep*\thebarcount)}
\let\barx\pgfmathresult

\filldraw[fill=#6, draw=none] (0,-\barx) rectangle (\barone,-\barx-\barwidth);
\filldraw[fill=#7, draw=none] (\barone, -\barx) rectangle (\barone+\bartwo,-\barx-\barwidth);
\filldraw[fill=#8, draw=none] (\barone+\bartwo,-\barx ) rectangle (\barone+\bartwo+\barthree,-\barx-\barwidth);

\node [label=180:\colorbox{#5}{\textcolor{#4}{##2}}] at (0,-\barx-0.175) {};
\addtocounter{barcount}{1}
}
\begin{tikzpicture}
\setcounter{barcount}{0}

}
{\end{tikzpicture}}

%----------------------------------------------------------------------------------------
% 	BUBBLE CHART
%----------------------------------------------------------------------------------------
\newcommand{\bubble}[5]{
\definecolor{tmpcol}{RGB}{50,50,#5}
% slice
\filldraw[fill=black,draw=none] (#1,0.5) circle (#3);

% outer label
\node[label=\textcolor{black}{#4}] at (#1,0.7) {};
}

\newcommand{\bubbles}[2]{
%reset counters
\setcounter{a}{0}
\setcounter{c}{150}
\begin{tikzpicture}[scale=3]
\foreach \p/\t in {#1} {
\addtocounter{a}{1}
\bubble{\thea/2}{\theb}{\p/25}{\t}{1\p0}
}
\end{tikzpicture}
}


%----------------------------------------------------------------------------------------
%	custom sections
%----------------------------------------------------------------------------------------

% create a coloured box with arrow and title as cv section headline
% param 1: section title
%
\newcommand{\cvsection}[1] {
\textcolor{white}{\MakeUppercase{\textbf{#1}}}
}

\newcommand{\cvsect}[1]{
\colorbox{black}{{\cvsection{#1}}}\\\\%
}

%----------------------------------------------------------------------------------------
% ENTRY LIST
%----------------------------------------------------------------------------------------
\usepackage{tabularx}

\setlength{\tabcolsep}{0pt}
\newenvironment{entrylist}{%
\begin{tabular*}{\textwidth}[t]{@{\extracolsep{\fill}}ll}
}{%
\end{tabular*}
}

\newcommand{\entry}[4]{%
\parbox[t]{3.5cm}{%
#1%
}%
&\parbox[t]{14cm}{%
\textbf{#2}%
\hfill%
{\footnotesize \textbf{\textcolor{black}{#3}}}\\%
#4%
}\\\\}

\newcommand{\entryy}[3]{%
\parbox[t]{17.5cm}{%
\textbf{#1}%
\hfill%
{\footnotesize \textbf{\textcolor{black}{#2}}}\\%
#3%
}\\\\}

\newcommand{\slashsep}{
\hspace{2mm}/\hspace{2mm}
}

%----------------------------------------------------------------------------------------
%	DOCUMENT CONTENT
%----------------------------------------------------------------------------------------
\graphicspath{{./assets/}}
\begin{document}

%----------------------------------------------------------------------------------------
%	TITLE HEADLINE
%----------------------------------------------------------------------------------------
\begin{minipage}[t]{0.25\textwidth}\hrule height 0pt width 0pt%
\includegraphics[scale=0.15]{avatar}
\end{minipage}
\begin{minipage}[t]{0.45\textwidth}\hrule height 0pt width 0pt%
\vspace{5mm}
\colorbox{black}{{\HUGE\textcolor{white}{\textbf{Dmitry}}}}%
\\\colorbox{black}{{\HUGE\textcolor{white}{\textbf{Bitsulia}}}}%
\end{minipage}%
\begin{minipage}[t]{0.3\textwidth}\hrule height 0pt width 0pt%
\vspace{2mm}
\small%
\icontext{MapMarker}{12}{Saint-Petersburg, Russia}\\
\icontext{Phone}{12}{+7 911 764-09-57}\\
\icontext{At}{12}{\href{mailto:gaudima@gmail.com}{gaudima@gmail.com}}\\
\icontext{Github}{12}{\href{https://github.com/gaudima}{github.com/gaudima}}\\
\end{minipage}%

% manage space by reducing font size
\small%
\vspace{0.5cm}

%----------------------------------------------------------------------------------------
%	SKILLS AND TECHNOLOGIES
%----------------------------------------------------------------------------------------

\cvsect{Who Am I?}%
\begin{minipage}[t]{0.4\textwidth}%
Software Developer, graduate of Saint-Petersburg ITMO University. Have a knowledge in areas such as:
Algorithms and Data Structures, Parsers, Theory of Computational Complexity, DBMS, Machine Learning and VCS (Git)\\
\end{minipage}%
\hfill
\begin{minipage}[t]{0.5\textwidth}\hrule height 0pt width 0pt%
\vspace{-10pt}%
\begin{barchart}{10}{5.5}{red}{white}{black}{black}{black}{black}
\baritem{50}{Java}{0}{0}{80}
\baritem{80}{Kotlin}{0}{0}{70}
\baritem{80}{C++}{0}{0}{100}
\baritem{80}{JavaScript}{0}{0}{70}
\baritem{40}{HTML}{0}{0}{60}
\baritem{50}{CSS}{0}{0}{50}
\end{barchart}
\end{minipage}%

\vspace{5mm}
\cvsect{Experience}
\begin{entrylist}
\entry
{08.2018 – now\\\footnotesize{part time}}
{C++ Developer}
{\href{http://amungo-navigation.com}{Amungo}}
{\vspace{-4mm}
\begin{itemize}[leftmargin=*]
\item Nut2nt+ receiver module for gnss-sdr.
\item Linux library for custom nut2nt+ board with GMSL interface for Nvidia Drive.
\item CUDA acquisition and tracking module for gnss-sdr.
\end{itemize}\\
\texttt{C}\slashsep\texttt{C++}\slashsep\texttt{CUDA}\slashsep\texttt{Linux}}
\entry
{08.2018 – now\\\footnotesize{part time}}
{Front-end Developer}
{\href{http://pocketpower.su}{PocketPower}}
{\vspace{-4mm}
\begin{itemize}[leftmargin=*]
\item Customized OTRS helpdesk software for service-workers.
\item Front-end side of application for power-bank rental.
\end{itemize}\\\\
\texttt{Perl}\slashsep\texttt{JS}\slashsep\texttt{Angular}\slashsep\texttt{Ionic}}
\end{entrylist}

\vspace{5mm}
\cvsect{Education}
\begin{entrylist}
\entry
{2018 – now}
{Master's Degree}
{ITMO University}
{Development of corporate information systems.}
\entry
{2014 – 2018}
{Bachelor's Degree}
{ITMO University}
{Applied mathematics and computer science.}
\end{entrylist}

\vspace{5mm}
\cvsect{Projects}
\begin{entrylist}
\entryy
{Metorck}
{\href{https://github.com/gaudima/metrock}{github.com/gaudima/metrock}}
{An app helping to find fastest routes in russian metro for pebble time watch.}
\entryy
{C-Boy}
{\href{https://github.com/gaudima/c-boy}{github.com/gaudima/c-boy}}
{Classic gameboy emulator written as a project for C++ course.}
\end{entrylist}

\vspace{5mm}
\begin{minipage}[t]{0.3\textwidth}\hrule height 0pt width 0pt%
\cvsect{Languages}
\textbf{Russian} - native\\
\textbf{English} - C1\\
\textbf{German} - A1
\end{minipage}%
\begin{minipage}[t]{0.4\textwidth}\hrule height 0pt width 0pt%
\cvsect{Achievements}
\textbf{IPChain hackathon} - 3rd place\\
\textbf{hackRussia hackathon} - 8th place
\end{minipage}%
\begin{minipage}[t]{0.3\textwidth}\hrule height 0pt width 0pt%
\cvsect{Hobbies}
Drawing, Snowboarding,\\Playing board games
\end{minipage}%

\end{document}